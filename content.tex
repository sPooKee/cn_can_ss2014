\section{CAN Bus}

\subsection{Was ist CAN?}
	
Das Controller Area Network ist ein serielles Bus System und wurde
1983, zur Vernetzung von Steuergeräten in Automobilen, von der Firma
Bosch GmbH entwickelt. Doch nicht nur in der Automobilindustrie,
sondern auch in der Medizintechnick, Flug- und Raumfahrttechnik sowie
in Aufzuganlagen und im Schiffsbau, kann man auf einen CAN Bus stoßen
\citep[nach][]{WI1}. Der Vorteil gegüber der herkömmlichen Verkablung
lag darin, das mehrere Knoten über dieselbe Leitung Kommunizieren
können. Alle Teilnehmer können Pakete priorisiert mit
Kollisionserkennung auf den Bus legen, hierbei wird die Nachricht,
nicht der Empfänger adressiert.
\\
Die physikalischen Eigenschaften und der Formale Aufbau des CAN Busses
wird in folgenden ISO Normen genau spezifiziert \citep[nach][]{WI1}:

\begin{itemize}
\item ISO 11898-1:2003 Road vehicles — Controller area network — 
Part 1: Data link layer and physical signalling
\item ISO 11898-2:2003 Road vehicles — Controller area network — 
Part 2: High-speed medium access unit
\item ISO 11898-3:2006 Road vehicles — Controller area network — 
Part 3: Low-speed, fault-tolerant, medium dependent interface
\item ISO 11898-4:2004 Road vehicles — Controller area network — 
Part 4: Time-triggered communication
\item ISO 11898-5:2007 Road vehicles — Controller area network — 
Part 5: High-speed medium access unit with low-power mode
\item ISO/NP 11898-6 Road vehicles — Controller area network — 
Part 6: High-speed medium access unit with selective wake-up functionality
\end{itemize}

\subsubsection{Übertragungsgeschwindigkeiten}

Aufgrund der Ausbreitungsgeschwindigkeit der Signale auf dem Bus, ist die 
Übetragungsgeschwindigkeit von der Leitungslänge abhängig. Die Tabelle 
\ref{tab:speed} beinhaltet eine Zuordnung von Leitungslängen und deren 
maximaler Übertragunsgeschwindigkeit und andersrum.

\begin{table}[h]
	\centering
	\begin{tabular}{lcr}
		Bus Speed & Bus Length \\
		\hline
		1 Mbit/s & 40m \\
		500 kbit/s & 100m \\
		125 kbit/s & 500m \\
		50 kbit/s & 1 km \\
	\end{tabular}
	\label{tab:speed}
	\caption{Übertragungsgeschwindigkeiten}
\end{table}

\subsection{Busankoplung/Vernetzung von CAN-Knoten}

Der CAN Bus arbeitet nach dem Multi-Master Prinzip, wonach es jedem Knoten 
möglich ist, mit jedem anderen Knoten zu kommunizieren. Üblicherweise in 
Linientopologie verbunden, können laut ISO 119898 bis zu 32 Knoten damit
verbunden werden. Die Signalübetragung erfolgt symmetrisch.

\begin{figure}[h] 
\centering
\includegraphics[width=0.7\textwidth]{figures/cannet}
\caption{Vernetzung von CAN Knoten \citep{VEC}} 
\label{data}
\end{figure} 

Auf dem physischen Medium werden die einzelnen Bits über Spannungsdifferenzen
abgebildet. Man unterscheidet hier zwischen Highspeed-CAN und Lowspeed-CAN.

\clearpage
\subsection{CAN-Bus Pegel} Beim CAN-Bus erfolgt die Übertragung der
Bits über 2 Pegel: 
\begin{itemize} 
\item Dominant 
\item Rezessiv
\end{itemize}

Wenn beide Pegel gleichzeitig gesendet werden, so überschreibt immer
der dominante Pegel den rezessiven Pegel.

\begin{figure}[h] 
\centering
\includegraphics[width=0.7\textwidth]{figures/highcan}
\caption{Highspeed-CAN Buspegel \citep{VEC}} 
\label{data}
\end{figure} 

\begin{figure}[h] 
\centering
\includegraphics[width=0.7\textwidth]{figures/lowcan}
\caption{Loewspeed-CAN Buspegel \citep{VEC}} 
\label{data}
\end{figure} 
	
\subsection{Bit Stuffing} 
Bit-Stuffing beim CAN-Bus bedeutet, dass
nach 5 Bits mit gleichem Pegel ein ">Stuff Bit"< mit inversem Pegel
eingefügt wird. Die Empfänger filtern diese Stuff-Bits, dann nach dem
gleichen Schema wieder heraus.
\\ Bit stuffing wird einerseits
eingesetzt, weil Bitfolgen mit mehr als 5 Bits mit gleichem Pegel für
Steuerungszwecke eingesetzt werden, und andererseits auch wegen der
NRZ-Kodierung. Das heißt dass der Pegel bei zB 4 rezessiven Bits
gleich bleibt. Wenn dann beispielsweise 10 gleiche Bits übertragen
werden, so kann der Empfänger möglicherweise nicht mehr unterscheiden,
ob jetzt 10 oder 11 Bits übertragen wurden.
	
\subsection{CAN Frames} Beim CAN-Bus gibt es 4 unterschiedliche Arten
von Frames: 
\begin{itemize} 
\item Daten-Frame 
\item Remote-Frame 
\item Error-Frame 
\item Overload-Frame 
\end{itemize} 

\subsubsection{Daten Frame} 
Die Daten-Frames dienen dem Transport von Daten und können bis
zu 8 Byte an Nutzdaten enthalten. Bei den Daten-Frames kann zwischen
Standard Format und Extended Format unterschieden werden, die sich
hauptsächlich in der Länge des Identifiers unterscheiden. Der Aufbau
eines Daten Frames kann aus Abbildung \ref{data} entnommen werden.
	
\begin{figure}[h] 
\centering
\includegraphics[width=0.7\textwidth]{figures/data-frame}
\caption{Aufbau eines Daten-Frames \citep{HYC}} 
\label{data}
\end{figure} 
%Quelle: http://marco.guardigli.it/2010/10/hacking-your-car.html
		
Gestarted wird ein Frame durch das \textbf{SOF} (start-of-frame)
bestehend aus einem dominanten Bit, das der Synchronisation aller CAN-Knoten dient. Darauf folgt gleich das
Arbitrationsfeld und dem Kontrollfeld. Abhängig vom Format bestehen
diese aus \textbf{Basis Identifier} (11 Bits), \textbf{Extended
Identifier} (18 Bit), einem \textbf{RTR}-Bit (remote transmission
request; bei Remote-Frames rezessiv) sowie einem \textbf{SRR}-Bit
(substitude-remote-request; rezessiv). Ist das \textbf{IDE}-Bit
(identifier extension) rezessiv, so handelt es sich um das Extended
Format, das somit immer Nachrang über dem Standard Format hat.\\ Im
Kontrollfeld befindet sich neben 1-2 reservierten (aktuell nicht
verwendeten) Bits der aus 4 Bits bestehende \textbf{DLC} (data length
code), der die Länge des nachfolgenden \textbf{Datenfeldes} angibt.\\
Wie bereits erwähnt können mit den Daten-Frames bis zu 8 Byte, also 64
Bit, übertragen werden. dies kann jedoch nur in einheiten von 8 Bits
geschehen. Anschließend an das Datenfeld ist das CRC-Feld (cyclic
redundancy check; Prüfsummenfeld) bestehend aus 16 Bit (15 Bit + 1
rezessives Delimiter-Bit). Des weiteren enthält ein Daten-Frame ein
\textbf{ACK}-Feld (Acknowledge). Dieses wird verwendet, um den Empfang
eines korrekten Frames zu bestätigen. Der Sender sendet dafür ein
rezessives Bit. Jeder Empfänger der keinen Fehler feststellen konnte,
setzt einen dominanten Pegel und überschreibt somit den rezessiven des
Senders. Im Falle einer negativen Quittierung (rezessiver Pegel) muss
der Knoten, der den Fehler erkannt hat, nach dem ACK-Delimiter einen
Error-Frame aussenden.\\Das Ende des Frames wird mit 7 rezessiven Bits
dem \textbf{EOF} (end of frame) angezeigt. Anschließend an den Frame
muss ein ITM-Package (intermission oder inter-frame-space) bestehend
aus mindestens 3 rezessiven Bits gesendet werden.
	
\subsubsection{Remote Frame} Mit Hilfe der Remote-Frames können
Daten-Frames von anderen Teilnehmern angefordert werden. Der Frame
unterscheidet sich vom Daten-Frame durch ein rezessives Bit im
">RTR"<-Slot, wodurch Remote-Frames im Falle einer Kollision immer
Nachrang gegenüber den Daten-Frames haben, und durch ein Fehlen des
Datenfeldes.
	
\subsubsection{Error Frame} Erkennt ein CAN-Knoten einen Fehler, so
sendet er einen Error-Frame an alle anderen CAN-Knoten im Netzwerk.
Bei diesen Frames wird das Bit-Stuffing bewusst ignoriert. Der
Error-Frame besteht aus 2 Feldern, den Error-Flags und dem
Error-Delimiter (8 rezessive Bits). Die Error-Flags sind abhängig vom
Modus in dem sich ein CAN-knoten befindet. Ist der Knoten im
Fehler-Status \textit{">error active"<} so setzt er die Error-Flags
auf 6 dominante Bits. Befindet er sich hingegen im Fehler-Status
\textit{">error passive"<} so sendet er 6 rezessive Bits.
	
\subsubsection{Overload Frame} Overload-Frames werden als Zwangspause
zwischen Daten- und Remote-Frames genutzt. Dabei hat ein
Overload-Frame das gleiche Format wie ein Active-Error-Frame. Wie in
Abbildung \ref{overload} ersichtlich ist, besteht der Overload-Frame
ebenfalls aus 2 Feldern: dem Overload-Flag (6 dominante Bits) und dem
Overload-Delimiter (8 rezessive Bits). Ein Overload-Frame kann jedoch
nur während eines Interframespaces gesendet werden. Dies ermöglicht
die Unterscheidung von den Error-Frames, die während der Übertragung
einer Nachricht gesendet werden, sobald eben ein Fehler erkannt wurde.

	
\begin{figure}[h] 
\centering
\includegraphics[width=0.7\textwidth]{figures/overload-frame}
\caption{Aufbau eines Overload-Frames \citep{CBM}} 
\label{overload}
\end{figure} 
% Quelle: http://rs232-rs485.blogspot.co.at/2009/11/can-bus-message-frames-overload.html
	
\subsection{Zeichenkodierung}
	
Auf der Busleitung werden die einzelnen Bits der Pakete über den 
Non-Return-to-Zero Code (NRZ) codiert. Hierbei hat jeder Wert eines Bit 
einen konstanten Zusatand auf der Leitung (siehe Abb. \ref{nrzcode}).

\begin{figure}[h] 
\centering
\includegraphics[width=0.7\textwidth]{figures/nrzcode}
\caption{NRZ-Code \citep{NRZ}} 
\label{nrzcode}
\end{figure} 

Probleme bei dieser Codierung können auftreten, wenn mehrere gleiche Symbole in Folge
gesendet werden. Abhilfe schafft hier Bit-Struffing, näheres dazu in Kapitel ~\ref{sec:security}.

Als Vergleich zur NRZ Codierung sollte man noch den Manchester-Code erwähnen, bei dem
jeweils eine steigende, bzw. fallende Taktflanke einen Bit-Wert repräsentiert (Abb. \ref{mancode}).
Ein Bitstuffing ist hier nicht nötig, da auch gleiche aufeinanderfolgende Bits durch eine Taktflanke 
reproduzierbar sind.

\begin{figure}[h] 
\centering
\includegraphics[width=0.7\textwidth]{figures/mancode}
\caption{Manchester-Code \citep{MAN}} 
\label{mancode}
\end{figure} 
	
\subsection{Synchronisatzion}

Da die CAN Knoten nur eine gemeinsame Baud-Rate, jedoch keinen gemeinsamen Takt besitzen, 
wird der Daten-Frame selbst zur Synchronization verwendet. Es gibt eine Hard- und eine Soft-Synchronization,
wobei die Hard-Synchronization die fallende Flanke des Startbits zur Synchronizierung verwendet. Alle 
nachfolgenden abfallenden Flanken werden zur Soft-Synchronizierung verwendet. Durch die Soft-Synchronizierung
kann sich der Takt nur noch in einem spezifizierten Rahmen verschieben \citep[nach][]{BSY}.
	
\subsection{Buszugriffsverfahren}
	
\subsection{Fehlermanagement}
	
\subsection{Sicherungsmechanismen und Fehlererkennung}
\label{sec:security}

\newpage
\section{CAN Bus over Ethernet}
	
\subsection{Probleme}
	
\subsection{CAN Schnittstelle}

CAN Schnittstellen erlauben es, eine CAN-Bus Struktur an Ethernet Topologien anzubinden. Dies ermöglicht anderen Netzwerkgeräten mit den Teilnehmern im CAN-Bus und vice versa zu kommunizieren. Typischerweise wird eine CAN Schnittstelle verwendet, um den CAN Bus über Ethernet mit einem PC zu verbinden. Dies ermöglicht einen flexiblen Zugriff auf die CAN-Systeme über LAN oder sogar Internet. Die CAN Schnittstelle verhält sich dabei wie ein normaler CAN Knoten. \citep{STE}

Die CAN Nachrichten werden vom CAN Gateway in Ethernet Packete eingepackt und dann über die Ethernet Verbindung gesendet. Dabei gehen jedoch wichtige Eigenschaften des CAN-Busses verloren. Zum Beispiel wird das ACK-Feld im CAN-Frame nicht mehr vom eigentlichen Empfänger, dem PC, sondern direkt von der CAN Schnittstelle gesetzt. Dadurch kann nicht sichergestellt werden, ob die Nachricht auch tatsächlich fehlerfrei beim PC angekommen ist.
Abbildung \ref{gateway} zeigt die Topologie des Netzwerkes bestehend aus CAN-Netzwerk, CAN-Schnittstelle und Ethernet.

\begin{figure}[h] 
\centering
\includegraphics[width=0.7\textwidth]{figures/can_gateway}
\caption{CAN-Schnittstelle} 
\label{gateway}
\end{figure} 

\subsection{CAN Ethernet Bridge}

Die CAN Ethernet Bridge ermöglicht es, mehrere CAN Netzwerke miteinander über Ethernet (LAN oder Internet) zu verbinden. Abbildung \ref{bridge} zeigt zwei CAN Busse, die über Ethernet miteinander verbunden sind. Die CAN Bridges verhalten sich hierbei wieder wie CAN Teilnehmer. Auch hier gehen wichtige Eigenschaften des CAN-Busses, wie beispielsweise das Setzen des ACK-Feldes durch den Empfänger oder das eigentliche Zeitverhalten im jeweiligen CAN-Bus, verloren. Denn das ACK-Feld wird direkt von der Bridge und nicht von den eigentlichen Empfängern gesetzt.

Vorteil einer solchen Entkopplung der CAN Netzwerke ist, dass eine Filterung der gesendeten Nachrichten möglich ist, und somit der Datenfluss zwischen den CAN Netzwerken beeinflussbar ist. So können beispielsweise Fehler abgefangen und begrenzt werden. Ein weiterer Vorteil einer Entkopplung ist, dass das Zeitverhalten der Busse getrennt betrachtet werden kann und für Analysen anstatt eines riesigen Busses nur einen Teil des Busses betrachtet werden muss.

\begin{figure}[h] 
\centering
\includegraphics[width=0.8\textwidth]{figures/can_bridge}
\caption{CAN Ethernet Bridge} 
\label{bridge}
\end{figure} 

\subsection{Transparente Ethernet Bridge}
	
\subsection{Bestehende Lösungen}



\newpage \addcontentsline{toc}{section}{Abbildungen} \listoffigures

\newpage \addcontentsline{toc}{section}{Literatur}
%\bibliographystyle{abbrv} 
\bibliographystyle{alphadin}
\bibliography{quellen}
